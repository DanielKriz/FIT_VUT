% proj2.tex
% Filip Kocica [xkocic01@fit.vutbr.cz]
% ITY - Sazba dokumentu s matematickými výrazy
% 7/3/2017

\documentclass[11pt, titlepage, a4paper, twocolumn]{article}
\usepackage[left=1.5cm,text={18cm, 25cm},top=2.5cm]{geometry}
\usepackage[czech]{babel}
\usepackage[utf8]{inputenc}
\usepackage{times}
\usepackage{amsthm,amssymb,amsmath,enumerate}

\theoremstyle{definition}
\newtheorem{definition}{Definice}[section]
\newtheorem{algorithm}[definition]{Algoritmus}
\theoremstyle{plain}
\newtheorem{sentence}{Věta}

\begin{document}

		\begin{titlepage}
    \begin{center}
		\Huge{\textsc{Fakulta informačních technologií\\
		Vysoké učení technické v~Brně}\\}
		\vspace{\stretch{0.382}}
		\LARGE{Typografie a~publikování -- 2. projekt\\
		Sazba dokumentů a~matematických výrazů\\}
		\vspace{\stretch{0.618}}
		\end{center}
		\Large{2017 \hfill Filip Kočica}
		\end{titlepage}

   % \setcounter{secnumdepth}{0}
    \section*{Úvod}
        V~této úloze si vyzkoušíme sazbu titulní strany, matematických vzorců, prostředí a~dalších textových
        struktur obvyklých pro technicky zaměřené texty napríklad rovnice (\ref{r1:rovnice_prvni}) nebo definice
        \ref{d1:definice_jedna} na straně \pageref{d1:definice_jedna}.

        Na titulní straně je využito sázení nadpisu podle optického středu s~využitím zlatého řezu. Tento postup
        byl probírán na přednášce.
    \par
    %\setcounter{secnumdepth}{1}
    \section{Matematický text}
        Nejprve se podíváme na sázení matematických symbolů a~výrazů v~plynulém textu. Pro množinu $V$ označuje
        card($V$) kardinalitu $V$. Pro množinu $V$ reprezentuje $V^*$ volný monoid generovaný množinou $V$
        s~operací konkatenace. Prvek identity ve volném monoidu $V^*$ znacíme symbolem $\epsilon$.
        Nechť $V^+$ = $V^*$ -- \{$\epsilon$\} Algebraicky je tedy $V^+$ volná pologrupa generovaná množinou $V$
        s~operací konkatenace. Konečnou neprázdnou množinu $V$ nazveme $abeceda$. Pro $w \in V^*$ označuje
				$|w|$ délku řetězce $w$. Pro $W \subseteq V$ označuje occur($w$,$W$) počet výskytů symbolu z~$W$
				v~řetězci $w$ a~sym($w$, $i$) určuje $i$-tý symbol řetězce $w$; například sym($abcd$,3) $=$ $c$.

        Nyní zkusíme sazbu definic a~vět s~využitím balíku \texttt{amsthm}.
    \par
    \begin{definition} \label{d1:definice_jedna}
        \emph{Bezkontextová gramatika} je čtveřice $G$ = ($V$,$T$,$P$,$S$), kde $V$ je totální abeceda,
        $T \subseteq V$ je abeceda terminálu, $S \in  (V - T)$ je startující symbol a~$P$ je konečná množina
        \emph{pravidel} tvaru $q: A \rightarrow \alpha$, kde $A \in (V - T)$, $\alpha \in V^*$ a~$q$ je návěští
        tohoto pravidla. Nechť $N = V - T$ značí abecedu neterminálu. Pokud $q: A \rightarrow \alpha \in P$,
        $\gamma, \delta \in V^*, G$ provádí derivační krok z~$\gamma A \delta$ do $\gamma \alpha \delta$ podle
        pravidla $q: A \rightarrow \alpha$, symbolicky píšeme $\gamma A \delta \Rightarrow \gamma \alpha \delta
        \left[q: A \rightarrow \alpha \right]$ nebo zjednodušene $\gamma A \delta \Rightarrow \gamma \alpha
        \delta$. Standardním způsobem definujeme $\Rightarrow ^m$, kde $m \geq 0$. Dále definujeme
        tranzitivní uzávěr $\Rightarrow ^+$ a~tranzitivně-reflexivní uzávěr $\Rightarrow ^*$.
    \end{definition}

    Algoritmus můžeme uvádět podobně jako definice textové, nebo využít pseudokódu vysázeného ve vhodném
    prostředí (napríklad \texttt{algorithm2e}).

    \begin{algorithm}
        \emph{Algoritmus pro ověření bezkontextovosti gramatiky. Mějme gramatiku $G = (N, T, P, S)$.}
        \begin{enumerate}[\itshape 1.]
        \item \label{p1:p} \emph{Pro každé pravidlo $p \in P$ proveď test, zda $p$ na levé straně obsahuje právě
        jeden symbol z~$N$.}
        \item \emph{Pokud všechna pravidla splňují podmínku z~kroku (\ref{p1:p}), tak je gramatika $G$
        bezkontextová.}
        \end{enumerate}
    \end{algorithm}
    \begin{definition}
        \emph{Jazyk} definovaný gramatikou $G$ definujeme jako $L(G) = \{w \in T^* | S \Rightarrow^* w\}$.
    \end{definition}
    \subsection{Podsekce obsahující větu}
        \begin{definition}
            Nechť $L$ je libovolný jazyk. $L$ je \emph{bezkontextový jazyk}, když a~jen když $L=L(G)$,
            kde $G$ je libovolná bezkontextová gramatika.
        \end{definition}
        \begin{definition}
            Množinu $\mathcal{L}_{CF}$ $=$ \{$L|L$ je bezkontextový jazyk\} nazýváme
            \emph{třídou bezkontextových jazyků.}
        \end{definition}
        \begin{sentence} \label{v1:veta}
            Nechť $L_{abc}=\{a^nb^nc^n|n \geq 0\}$. Platí, že $L_{abc} \notin \mathcal{L}_{CF}$.
        \end{sentence}
				\begin{proof}
        		Důkaz se provede pomocí Pumping lemma pro bezkontextové jazyky, kdy ukážeme,
        		že není možné, aby platilo, což bude implikovat pravdivost věty \ref{v1:veta}.
    		\end{proof}
		\par

    \section{Rovnice a~odkazy}
        Složitější matematické formulace sázíme mimo plynulý text. Lze umístit několik výrazů na jeden řádek, ale
        pak je třeba tyto vhodně oddělit, například příkazem \verb|\quad|.

        $$\sqrt[x^2]{y_0^3}\quad \mathbb{N}=\{0,1,2,\ldots\}\quad x^{y^y} \neq x^{yy}\quad
        z_{i_j}\not\equiv z_{ij}$$

        V~rovnici (\ref{r1:rovnice_prvni}) jsou využity tři typy závorek s~různou explicitně definovanou velikostí.

        \begin{eqnarray} \label{r1:rovnice_prvni}
        \left\{\left[\left(a + b\right) * c\right]^d + 1\right\}&=&x\\
        \lim_{x \rightarrow \infty} \frac{\sin^2x + \cos^2x}{4}&=&y \nonumber
        \end{eqnarray}

        V~této větě vidíme, jak vypadá implicitní vysázení limity $\lim_{n \rightarrow \infty} f(n)$  v~normálním
        odstavci textu. Podobně je to i~s~dalšími symboly jako $\sum_1^n$ ci $\bigcup_{A \in B}$. V~případě vzorce
        $\lim\limits_{x \rightarrow 0} \frac{\sin x}{x} = 1$ jsme si vynutili méně úspornou sazbu príkazem
        \verb|\limits|.

        \begin{eqnarray}
        \int\limits_{a}^{b}f(x)dx&=&-\int_{b}^{a}f(x)dx\\
        \left(\sqrt[5]{x^4}\right)'=\left(x^{\frac{4}{5}}\right)'&=&\frac{4}{5}x^{-\frac{1}{5}}=\frac{4}{5\sqrt[5]
        {x}}\\
        \overline{\overline {A \vee B}}&=&\overline{\overline A \wedge \overline B}
        \end{eqnarray}
    \par

    \section{Matice}
        Pro sázení matic se velmi často používá prostredí \texttt{array} a~závorky (\verb|\left|, \verb|\right|).

        $$\left(\begin{array}{cc} a+b&b-a\\
        \widehat{\xi+\omega}&\hat{\pi}\\
        \vec{a}&\overleftrightarrow{AC}\\
        0&\beta\\
        \end{array}\right)$$
        $$\textbf{A =}\left|\left| \begin{array}{cccc}
        a_{11} & a_{12} & \dots& a_{1n} \\
        a_{21}& a_{22} & \dots & a_{2n} \\
        \vdots & \vdots & \ddots & \vdots \\
        a_{m1} & a_{m2} & \dots & a_{mn}
        \end{array}\right|\right|$$
        $$\left|\begin{array}{cc}
        t &u\\
        v&w
        \end{array}\right|=tw-uv$$

        Prostredí \texttt{array} lze úspešne použít i~jinde.
        $$\binom{n}{k} = \begin{cases}
        \ \frac{n!}{k!(n-k!)} & \text{pro } 0\leq k \leq n\\
        \ 0& \text{pro } k < 0 \text{ nebo } k > n \end{cases}$$
    \par

    \section{Závěrem}
        V~případě, že budete potrěbovat vyjádřit matematickou konstrukci nebo symbol a~nebude se Vám dařit
        jej nalézt v~samotném \LaTeX u, doporučuji prostudovat možnosti balíku maker \AmS-\LaTeX.
        Analogická poučka platí obecně pro jakoukoli konstrukci v~\TeX u.
    \par

\end{document}
