% ITY 4. projekt - Filip Kocica [xkocic01@fit.vutbr.cz]
% 4.4.2017

\documentclass[11pt,a4paper,titlepage]{article}
\usepackage[left=2cm,text={17cm,24cm},top=3cm]{geometry}
\usepackage[T1]{fontenc}
\usepackage[czech]{babel}
\usepackage[utf8]{inputenc}

\bibliographystyle{czplain}

\begin{document}

		\begin{titlepage}
    \begin{center}
		\textsc{\Huge{Vysoké učení technické v~Brně\\}
		\huge{Fakulta informačních technologií\\}}
		\vspace{\stretch{0.382}}
		\LARGE{Typografie a~publikování -- 4. projekt\\}
		\Huge{Bibliografické citace\\}
		\vspace{\stretch{0.618}}
		\end{center}
		\Large{4.dubna 2017 \hfill Filip Kočica}
		\end{titlepage}

    \section{Motivace}
    Datové modelování je jednou z disciplín softwarového inženýrství. Je to proces, při němž
    se definují a analyzují požadavky na strukturu dat, s nimiž pracuje informační systém.
    Výsledkem tohoto procesu je datový model \cite{Wikipedia:Dat_mod}.

    UML, Unified Modeling Language je v softwarovém inženýrství grafický jazyk pro vizualizaci,
    specifikaci, navrhování a dokumentaci programových systémů. UML nabízí standardní způsob
    zápisu jak návrhů systému včetně konceptuálních prvků jako jsou business procesy a systémové
    funkce, tak konkrétních prvků jako jsou příkazy programovacího jazyka, databázová schémata~a
    znovupoužitelné programové komponenty.

    UML podporuje objektově orientovaný přístup k analýze, návrhu a popisu programových systémů.
    UML neobsahuje způsob, jak se má používat, ani neobsahuje metodiku(y), jak analyzovat,
    specifikovat či navrhovat programové systémy \cite{Wikipedia:UML}.

    \section{Návrh systému}
    Návrh systému je popis struktury systému, dat (návrh struktury databáze), rozhraní
    komponent, uživatelského rozhraní a případně i použitých algoritmů. Návrh by měl začínat
    tam, kde končí analýza. Avšak analýza je modelování bez úvah o implementaci a návrh už bere
    v potaz platformu, na níž bude systém implementován \cite{Fiala:IS}.

    Pro příznivce kvalitní literatury existuje mnoho publikací zabývající se návrhem systému~a
    datovým modelováním
    \cite{Merunka:Datove_modelovani}
    \cite{Lal:Graph_data_modeling}.


    \section{Generalizace}
    Generalizace neboli specializace nastává při dědění třídy. V Javě se pro tento vztah
    používá klíčové slovo extends. Třída tímto zdědí metody a atributy třídy, od které
    je děděno. Generalizace je jednosměrná, takže vytváří hierarchii tříd rodič -> potomek
    \cite{Bilek:Java_AST}.

    Generalizaci zakreslujeme jako plnou čáru, zakončenou na jedné straně prázdnou
    uzavřenou šipkou (nebo chcete-li trojúhelníkem). Šipka je na straně entity, ze které se dědí
    \cite{ITnetwork:Java_generalizace}.

    \section{Zahraniční články o matematice}
    Since the sturcuture of mathematical equations can be represented in the form of tree,
    mathematical equation similarity may be defined using tree similarity
    \cite{Clanek:Informatica}.
    The input systems were randomly generated by generator. The inputs are labeled
    dense and sparse, i.e. the proportion of zero coefficients is 12.5\%. The solvers use the
    fload data type for all the computation
    \cite{Clanek:CaI}.

    \section{Závěr}
    Tento text byl vysázen v \LaTeX u v rámci předmětu typografie a publikování a~to
    zejména díku panu doktoru B.~Křenovi, který předmět na FIT VUT zavedl
    \cite{Sbornik:PhD_Krena}.

    \newpage
    \bibliography{proj4}

\end{document}
